% !TeX root = main.tex
%!TeX spellcheck = en-US
\documentclass[11pt]{article}
\usepackage{amsmath,amssymb}
\usepackage{supertabular} 
% \usepackage[  
%      includeheadfoot, head=13pt, foot=2pc,
%      paperwidth=6.75in, paperheight=10in,
%      top=58pt, bottom=44pt, inner=46pt, outer=46pt,
%      marginparwidth=2pc,heightrounded
% ]{geometry}
\usepackage{geometry}
\usepackage{ifthen} 
% \usepackage{pdflscape} 
% \usepackage{alltt}%hack 
% \geometry{a4paper,dvips,twoside,left=22.5mm,right=22.5mm,top=20mm,bottom=30mm}
\usepackage{color}    
\usepackage{mathpartir}   
\usepackage{stmaryrd}   

\usepackage{libertine}
\usepackage{inconsolata}
% \usepackage{libertinust1math}

\usepackage{keyval}
\usepackage{ifthen} 
\usepackage{enumitem}   

\usepackage{amsthm}
\usepackage{hyperref} 

\newtheorem{theorem}{Theorem}[section]
\newtheorem{corollary}{Corollary}[section]
\newtheorem{lemma}{Lemma}[section]
\newcommand*{\lemmaautorefname}{Lemma}

\newcommand{\abe}{\ensuremath{\alpha\beta\eta}}

\usepackage[implicitPremiseBreaks]{ottalt}  
\inputott{GTFL_defns}       
 
\newcommand{\rrule}[1]{\rref*{#1}}


\begin{document}

\section{Source Language Syntax}

\begin{figure}
	$[[n]] \in \mathbb{Z}$, $[[b]]\in\mathbb{B}$

	\nonterms{e}
	\nonterms{O}
	\nonterms{v}
	\nonterms{r}
	\caption{Source Language Syntax: Terms}
	\label{fig:term-syntax}
\end{figure}

\begin{figure}

	\nonterms{T}
	\nonterms{ep}
	\nonterms{Gamma}

	\caption{Source Language Syntax: Types}
	\label{fig:type-syntax}
\end{figure}

\section{Source Language Statics and Dynamics}

\begin{figure}
	\ottdefnHastype{}
	\caption{Source Language: Type Rules}
	\label{fig:source-typerules}
\end{figure}

\begin{figure}
	\ottdefnConsistent{}
	\ottdefnMeet{}
	\caption{Source Language: Type Consistency and Precision}
	\label{fig:source-precision}
\end{figure}

\begin{figure}
	\ottdefnsSemantics{}
	\caption{Source Language: Small-Step Operational Semantics}
	\label{fig:source-semantics}
\end{figure}

\section{The Target Language}

\begin{figure}
	\nonterms{u}
	\nonterms{d}
	\nonterms{t}
	\caption{Target Language: Syntax}
	\label{fig:target-syntax}
\end{figure}

\section{The Translation}

\subsection{Translating Evidence}

\subsubsection{Helper Functions}

With our evidence represented as tuples with integer tags, we must represent the partial functions
on types in our target language.
The implementation is given in \autoref{fig:meet-impl}.
Doing this is straightforward: if one argument is $[[?]]$,
then we return the other argument. Otherwise, we check if we have simple or complex types.
For simple types, either $[[Bool /\ Bool ==== Bool]]$, $[[Nat /\ Nat ==== Nat]]$.
For complex types, we check that the tags agree, then recursively compute the meets of the sub-components.
If neither argument is $[[?]]$, and there is a tag mismatch, then we must raise an exception, retuning the $[[error]]$
continuation.

For the partial functions decomposing types, we first check if the input is $[[?]]$,
in which case we return $[[?]]$. Otherwise, we check the tag, and if it is correct, we return
the relevant sub-component of the type. In all other cases, we throw an error.
We give an example implementation for $\ottkw{dom}$ in \autoref{fig:dom-impl}: either we are given $[[?]]$
and return $[[?]]$, we are given $[[T1 -> T2]]$ and we return $[[T1]]$, or we raise an exception.
We omit $\ottkw{cod}$, $\ottkw{proj}_1$ and $\ottkw{proj}_2$, but they are implemented similarly.

\begin{figure}
	\begin{align*}
		 & [[MEET]] & = & [[fix self \ ty1 ty2 c . let tag1 := pi1 ty1 in let sub1 := pi2 ty1 in let isDyn1 := tag1 == DYN in  \\ ]]
		 &          &   & [[ if isDyn1 c[ty2]                                                                                  \\ ]]
		 &          &   & [[ let tag2 := pi1 ty2 in let sub2 := pi2 ty2 in let isDyn2 := tag2 == DYN in                        \\  ]]
		 &          &   & [[ if isDyn2 c[ty1]                                                                                  \\ ]]
		 &          &   & [[ let isNat1 := tag1 == NAT in let isNat2 := tag2 == NAT in                                         \\ ]]
		 &          &   & [[ if isNat1 (if isNat2 k[NAT] error)                                                                \\ ]]
		 &          &   & [[ let isBool1 := tag1 == BOOL in let isBool2 := tag2 == BOOL in                                     \\ ]]
		 &          &   & [[ if isBool1 (if isBool2 k[BOOL] error)                                                             \\ ]]
		 &          &   & [[ let isArrow1 := tag1 == ARR in let isArrow2 := tag2 == ARR in                                     \\ ]]
		 &          &   & [[ iif isArrow1                                                                                      \\ ]]
		 &          &   & [[ -- let dom1 := pi1 sub1 in let cod1 := pi2 sub1 in \\]]
		 &          &   & [[ -- iif isArrow2                                                                                   \\ ]]
		 &          &   & [[ -- -- let dom2 := pi1 sub2 in let cod2 := pi2 sub2 in \\]]
		 &          &   & [[ -- -- self [dom1, dom2, (\ meet1 . self[cod1, cod2, (\ meet2 . k[(ARR, (meet1, meet2))] )] )] ]]  \\
		 &          &   & [[ -- else error]]                                                                                   \\
		 &          &   & [[ let isProduct1 := tag1 == PROD in let isProduct2 := tag2 == PROD in                               \\ ]]
		 &          &   & [[ iif isProduct1                                                                                    \\ ]]
		 &          &   & [[ -- let lhs1 := pi1 sub1 in let rhs1 := pi2 sub1 in \\]]
		 &          &   & [[ -- iif isProduct2                                                                                 \\ ]]
		 &          &   & [[ -- -- let lhs2 := pi1 sub2 in let rhs2 := pi2 sub2 in \\]]
		 &          &   & [[ -- -- self [lhs1, lhs2, (\ meet1 . self[rhs1, rhs2, (\ meet2 . k[(PROD, (meet1, meet2))] )] )] ]] \\
		 &          &   & [[ -- else error]]                                                                                   \\
		 &          &   & [[ else error]]
		%
	\end{align*}
	\caption{CPS implementation of meet}
	\label{fig:meet-impl}
\end{figure}

\begin{figure}
	\begin{align*}
		 & [[DOM]] & = & [[\ ty c . let tag := pi1 ty1 in let sub := pi2 ty1 in let isDyn := tag == DYN in \\ ]]
		 &         &   & [[ if isDyn c[(DYN,0)]                                                            \\ ]]
		 &         &   & [[let isArrow := tag == ARR in                                                    \\ ]]
		 &         &   & [[if isArrow (let ret := pi1 sub in k[ret] ) error ]]
		%
	\end{align*}
	\caption{CPS implementation of domain}
	\label{fig:dom-impl}
\end{figure}

\begin{figure}
	\ottdefnTransform
	\caption{Translation: Terms and Programs}
	\label{fig:trans-terms}
\end{figure}

\begin{figure}
	\ottdefnEvTransform
	\caption{Translation: Evidence}
	\label{fig:trans-types}
\end{figure}

\begin{figure}
	\ottdefnValTransform
	\caption{Translation: Evidence}
	\label{fig:trans-values}
\end{figure}

\section{Correctness}

\begin{lemma}[Correctness of Evidence Translation]
	\label{lem:ev-correct}
	Consider evidence $[[ep]],[[ep']]$. Then, for any $[[k]]$:
	\begin{itemize}
		\item $[[MEET[ [|ep|], [|ep'|], k ] -->* k[ [| ep /\ ep' |] ] ]]$ if $[[ep /\ ep']]$ is defined.
		\item If $[[ep /\ ep']]$ is undefined, then $[[MEET[ [|ep|], [|ep'|] ] -->* error ]]$.
	\end{itemize}
	The same property holds for $[[dom ep]]$, $[[cod ep]]$, and $[[Proj@i ep]]$.
\end{lemma}

\begin{lemma}[Canonical Forms for Translated Values]
	\label{lem:canonical-trans}
	For an irreducible $[[v]]$, $[[ [|v|] ==== ( [|ep|], u) ]] $ for some evidence $[[ep]]$ and CPS-value $[[u]]$.
	Moreover, if $[[v]]$ is a raw irreducible, then $[[ep]]=[[<<?>>]]$.
\end{lemma}
\begin{proof}
	By inversion on the definition of $[[ [|v|] ]]$.
\end{proof}

\begin{lemma}[Value and Expression Translations Match]
	\label{lem:value-expr-trans}
	Let $[[v]]$ be an irreducible term. Then, for any $[[k]]$, $[[v]]$, $[[ [|v|]k -->* k [ [|v|] ]  ]]$.
\end{lemma}
\begin{proof}
	By induction on $[[v]]$. 

	\begin{itemize}
		\item Case $[[v]] = [[b]]$, $[[v]] = [[n]]$, or $[[v]] = [[\ x : T . e]]$: trivial.
		\item Case $[[v]] = [[(v1, v2)]]$.
		      % By our hypothesis, for any $[[k0]]$, $[[ [|v1|]k0 -->* k0 [ [|v1|] ]  ]]$ and $[[ [|v2|]k0 -->* k0 [ [|v2|] ]  ]]$.
		      So $[[ [|(v1, v2)|]k ==== [| v1 |](\ x1 . [|v2|](\ x2 . k[(DYN, (x1, x2))] ) )  ]]$,
		      which, by our hypothesis, reduces to  $[[t1 -->* (\ x1 . (\ x2 . k[(DYN, (x1, x2))] )[ [|v2|] ] )[ [|v1|] ] ]]$,
		      which we can then reduce to $[[k[(DYN, ([|v1|],[|v2|]))] ]]$.
		\item Case $[[v]] = [[ep r]]$. Since all raw irreducibles are themselves irreducible,
		      our inductive hypothesis gives that
		      $[[ [|r|] (\ x . let x1 := pi1 x in let x2 := pi2 x in MEET [ [|ep|],x1, (\y . k [(y,x2)]) ] )]]$
		      steps to $[[(\ x . let x1 := pi1 x in let x2 := pi2 x in MEET [ [|ep|],x1, (\y . k [(y,x2)]) ] )[ [|r|] ] ]]$.
		      By \autoref{lem:canonical-trans}, $[[ [|r|] ]]$ is of the form
		      $[[(DYN, u)]]$ for some $[[u]]$.
		      So we can then $\beta$-reduce and apply the let-substitutions to reach $[[ MEET[ [|ep|], DYN, u ]  ]]$.
		      By \autoref{lem:ev-correct}, this steps to $[[([|ep|], u)]]$.
		      By the rule \rrule{TransformEv}, this means that $[[ [|ep r|] ]]$ also steps to this value.


		      TODO 
	\end{itemize}
\end{proof}

\begin{lemma}[Translation Commutes With Substitution]
	\label{lem:subst-commut}
	$[[ [|[x |=> v]e|]k -->* [x |=> [|v|] ][|e|]k   ]]$.
\end{lemma}
\begin{proof}
	Follows from straightforward induction on $[[e]]$, combined with \autoref{lem:value-expr-trans}
	for the case where $[[e]]=[[x]]$.
\end{proof}

\begin{theorem}[Weak Simulation]

	If $[[e1 --> e2]]$, then for all $[[k]]$, $[[ [|e1|]k]] \equiv  [[ [|e2|]k  ]]$.

\end{theorem}
\begin{proof}
	We perform induction on the derivation tree of $[[e1 --> e2]]$.

	\begin{itemize}
		\item
		      \rrule{RedIfTrue}: then $[[e1]]=[[if true then e2 else e3]]$.
		      The translation $[[ [|true|]k' ==== k' [(DYN, true)]  ]]$ for any $[[k']]$,
		      so $[[ [|if true then e2 else e3|]k ]]$ is
		      $[[(\ x0 . let x := pi2 x0 in if x ([|e2|]k) ([|e3|]k) ) [(DYN, true)]  ]]$.
		      We can $\beta$-reduce to get $[[let x := pi2 (DYN, true) in if x [|e2|]k [|e3|]k ]]$,
		      and we can substitute $[[true]]$ for $[[x]]$ and reduce the $\ottkw{if}$ to get $[[ [|e2|]k ]]$.

		\item \rrule{RedIfFalse}: symmetric to RedIfTrue

		\item \rrule{RedIfEv}: $[[e1]]=[[if ep b then e2' else e3']]$.
		      We know that $[[ [| b|]k' ==== k' [(DYN, b)] ]]$,
		      so $[[ [| ep b |]k'' ==== (\x. let x1 := pi1 x in let x2 := pi2 x in MEET[ [|ep|], x1, (\y.k''[(y,x2)]) ] )[(DYN,b)] ]]$.
		      We can $\beta$-reduce, and substitute with the let-expressions, to get
		      $[[(\x.  MEET[ [|ep|], DYN, (\y.k''[(y,b)]) ] )]]$.
		      However, $[[ ep /\ <<?>>]] = [[<<?>>]]$, so by \autoref{lem:ev-correct} this steps to $[[k'' [([|ep|], b)] ]]$.
		      Since the translation of $\ottkw{if}$ ignores any evidence in the condition,
		      we can use the same reasoning from RedIfTrue to show that it steps to $[[e2]]$ if $[[b]]$ is true, and $[[e3]]$ if $[[b]]$ is false.
 
		\item \rrule{RedApp}: then $[[e1]] = [[ (\ x : T . e') v ]]$ and $[[e2]]=[[ [x |=> v]e' ]]$.
		Let $[[( [|ep|], u)]] = [[ [|v|] ]]$ (by \autoref{lem:canonical-trans}).
        If we apply \autoref{lem:value-expr-trans}, we can see that $[[ [|(\ x : T . e') v|]k ]]$
        steps to \\$[[ (\ x1  x2. let y1 := pi1 x1 in ,,,)[(DYN, (\ x c . [|e'|]c )), ( [|ep|], u) ]  ]]$.
        We can $\beta$-reduce and apply the let-substitutions to then step to
        \\$[[ DOM [DYN, \y1' . COD [ DYN, \y1''. MEET [y1', [|ep|], (\ y3 . (\ x c . [|e'|]c ) [(y3, u), (\z3 . let z3' := pi1 z3 in let z3'' := pi2 z3 in MEET[y1'', z3', (\z4. k[(z4, z3'')] ) ] ) ] )] ] ] ]]$.
        By applying \autoref{lem:ev-correct} for $[[DOM]], [[COD]]$ and $[[MEET]]$ of $[[?]]$ respectively,
        we can step to  
        \\$[[  (\ x c . [|e'|]c ) [([|ep|], u), (\z3 . let z3' := pi1 z3 in let z3'' := pi2 z3 in MEET[DYN, z3', (\z4. k[(z4, z3'')] ) ] ) ] ]]$.
		This then $\beta$-reduces to 
		\\$[[  [ x |=> ([|ep|], u) ][|e'|](\z3 . let z3' := pi1 z3 in let z3'' := pi2 z3 in MEET[DYN, z3', (\z4. k[(z4, z3'')] ) ] ) ]]$.
		But, then, by \autoref{lem:ev-correct} and $\eta$-equivalence, this is equivalent to
		$[[  [ x |=> ([|ep|], u) ][|e'|]k  ]]$.
		But we know that this is $[[  [ x |=> [|v|] ][|e'|]k  ]]$ 
		Finally, \autoref{lem:subst-commut} gives us that
		this is equivalent to $[[ [| [x |=> v]e' |]k  ]]$.

		\item \rrule{RedAppEv}: then $[[e1]] = [[ ep1 (\ x : T . e') ep2 v ]]$ and $[[e2]]=[[ cod ep1 ([x |=> (dom ep1 /\ ep2) v]e') ]]$.
		Let $[[( [|ep2|], u)]] = [[ [|v|] ]]$ (by \autoref{lem:canonical-trans}).
        If we apply \autoref{lem:value-expr-trans}, we can see that $[[  [|ep1 (\ x : T . e') ep2 v|]k ]]$
        steps to \\$[[ (\ x1  x2. let y1 := pi1 x1 in ,,,)[( [|ep1|] , (\ x c . [|e'|]c )), ( [|ep2|], u) ]  ]]$.
        We can $\beta$-reduce and apply the let-substitutions to then step to
        \\$[[ DOM [ [|ep1|] , \y1' . COD [  [|ep1|] , \y1''. MEET [y1', [|ep2|], (\ y3 . (\ x c . [|e'|]c ) [(y3, u), (\z3 . let z3' := pi1 z3 in let z3'' := pi2 z3 in MEET[y1'', z3', (\z4. k[(z4, z3'')] ) ] ) ] )] ] ] ]]$.
        By applying \autoref{lem:ev-correct} for $[[DOM]], [[COD]]$ and $[[MEET]]$ of $[[?]]$ respectively,
        we can step to  
        \\$[[  (\ x c . [|e'|]c ) [([|dom ep1 /\ep2|], u), (\z3 . let z3' := pi1 z3 in let z3'' := pi2 z3 in MEET[ [|cod ep1|] , z3', (\z4. k[(z4, z3'')] ) ] ) ] ]]$.
		This then $\beta$-reduces to 
		\\$[[  [ x |=> ([|dom ep1 /\ep2|], u) ][|e'|](\z3 . let z3' := pi1 z3 in let z3'' := pi2 z3 in MEET[ [|cod ep1|] , z3', (\z4. k[(z4, z3'')] ) ] ) ]]$.
		% But, then, by \autoref{lem:ev-correct} and $\eta$-equivalence, this is equivalent to
		% $[[  [ x |=> ([|dom ep1 /\ep2|], u) ][|e'|]k  ]]$.
		But, by the rule \rrule{TransformEv}, this is $\alpha$-equivalent to  $[[ [|cod ep1 ([x |=> (dom ep1 /\ ep2) v]e')|]k ]]$,
		giving us our result.
		 
	\end{itemize}
\end{proof}
 
\section{Incorrectness}

\end{document}
